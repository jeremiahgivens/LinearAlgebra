\documentclass[10pt,a4paper]{article}
\usepackage[utf8]{inputenc}
\usepackage[a4paper,%
            left=.75in,right=.75in,top=1in,bottom=1in]{geometry}
\setlength{\headsep}{0.25in}

\usepackage{amsthm}

\usepackage{graphicx}
\usepackage{pgfplots}
            
\usepackage[english]{babel}

\newtheorem{theorem}{Theorem}
\newtheorem{lemma}{Lemma}
\newtheorem{corollary}{Corollary}
\newtheorem{case}{Case}

\usepackage{amsthm}
\usepackage{lipsum}
\usepackage{tikz}

\makeatletter
\newcommand{\proofpart}[2]{%
  \par
  \addvspace{\medskipamount}%
  \noindent\emph{Part #1: #2}\par\nobreak
  \addvspace{\smallskipamount}%
  \@afterheading
}
\makeatother

\newcommand\restr[2]{{% we make the whole thing an ordinary symbol
  \left.\kern-\nulldelimiterspace % automatically resize the bar with \right
  #1 % the function
  \vphantom{\big|} % pretend it's a little taller at normal size
  \right|_{#2} % this is the delimiter
  }}

\theoremstyle{definition}
\newtheorem{definition}{Definition}
\newtheorem{remark}{Remark}

\usepackage{mathtools}
\DeclarePairedDelimiter\bra{\langle}{\rvert}
\DeclarePairedDelimiter\ket{\lvert}{\rangle}
\DeclarePairedDelimiterX\braket[2]{\langle}{\rangle}{#1 \delimsize\vert #2}

\usepackage{amsmath}
\usepackage{amsfonts}
\usepackage{amssymb}
\usepackage{fancyhdr}
\usepackage{tkz-euclide}

\DeclareMathOperator{\interior}{int}

\newcommand{\Tau}{\mathcal{T}}
\newcommand{\F}{\mathbb{F}}
\newcommand{\R}{\mathbb{R}}
\newcommand{\C}{\mathbb{C}}
\newcommand{\B}{\mathcal{B}}
\DeclarePairedDelimiterX{\iprod}[1]{\langle}{\rangle}{#1}

\newenvironment{amatrix}[1]{%
  \left(\begin{array}{@{}*{#1}{c}|c@{}}
}{%
  \end{array}\right)
}

\usepackage{calligra}
\DeclareMathAlphabet{\mathcalligra}{T1}{calligra}{m}{n}
\DeclareFontShape{T1}{calligra}{m}{n}{<->s*[2.2]callig15}{}

\newcommand{\scripty}[1]{\ensuremath{\mathcalligra{#1}}}

\pagestyle{fancy}
\author{Jeremiah Givens}
\newcommand{\subject}{Linear Algebra}
\newcommand{\Date}{9/2/2021} 
\makeatletter
\rhead{{\small\@author}}
\lhead{{\small\subject}}
\chead{{\large Homework 6}}
\cfoot{}
\rfoot{\thepage}
\lfoot{\today}

\renewcommand{\theequation}{\arabic{equation}}

\begin{document}
\section*{Problem 1}
\begin{theorem}
Let $x,y$ be elements of the inner product space $V$. Then
\begin{align*}
||x + y ||^2 + ||x - y ||^2 = 2||x||^2 + 2||y||^2.
\end{align*}
\end{theorem}

\subsection*{Solution}
\begin{proof}
Using the axioms of an inner product space and theorem 6.1, we have
\begin{align*}
||x + y ||^2 &= \langle x + y , x + y \rangle \\
&= \langle x , x + y \rangle + \langle y , x + y \rangle\\
&= \langle x , x \rangle + \langle x , y \rangle + \langle y , y \rangle + \langle y , x \rangle\\
&= ||x||^2 + ||y||^2 + \langle x , y \rangle + \langle y , x \rangle.
\end{align*}
Similarly,
\begin{align*}
||x - y ||^2 &= \langle x - y , x - y \rangle \\
&= \langle x , x - y \rangle - \langle y , x - y \rangle\\
&= \langle x , x \rangle - \langle x , y \rangle - \langle y , x \rangle + \langle y , y \rangle\\
&= ||x||^2 + ||y||^2 - \langle x , y \rangle - \langle y , x \rangle.
\end{align*}
Thus, we have
\begin{align*}
||x + y ||^2 + ||x - y ||^2 = 2||x||^2 + 2||y||^2,
\end{align*}
as desired.
\end{proof}

\section*{Problem 2}
Let $V$ be a real or complex finite dimensional vector space, and let $\B = \{v_1, v_2, \ldots, v_k\}$ be a basis for $V$. Then for any $x, y \in V$, we may write
\[x = \sum_i a_i v_i \quad \text{and} \quad y = \sum_j b_j v_j.\]

Define 
\[\iprod{x, y}= \sum_{i =1}^k a_i \overline{b_i}.\]
Prove that $\iprod{\cdot,\cdot}$ is an inner product on $V$ and that $\B$ is an orthonormal basis for $V$. Conclude that every real or complex vector space may be regarded as an inner product space.

\subsection*{Solution}
\begin{proof}
Two show properties (a) and (b), let $z = \sum_i c_i v_i$ and let $c \in \F$. Then, we have
\begin{align*}
\iprod{cx + z, y} &= \sum_{i =1}^k (ca_i + c_i)\overline{b_i}\\
&= c\sum_{i =1}^k a_i\overline{b_i} + \sum_{i =1}^k c_i\overline{b_i}\\
&= c \iprod{x, y} + \iprod{z, y}.
\end{align*}
For property $c$, we have 
\begin{align*}
\overline{\iprod{x, y}} &= \overline{\sum_{i =1}^k a_i \overline{b_i}}\\
&= \sum_{i =1}^k \overline{a_i \overline{b_i}}\\
&= \sum_{i =1}^k \overline{a_i} b_i\\
&= \sum_{i =1}^k b_i \overline{a_i}\\
&= \iprod{y, x}.
\end{align*}
Finally, to verify condition (d), let $x \not = 0$. We have
\begin{align*}
\iprod{x,x} &= \sum_{i =1}^k a_i \overline{a_i}\\
&= \sum_{i =1}^k |a_i|^2.
\end{align*}
Now, since each term is a complex square, this is a sum of nonnegative real numbers, and is therefore real and nonnegative. Furthermore, since $x \not = 0$, this sum has at least one term that is nonzero. Thus, this sum is a positive real number, and we have shown this function is an inner product on $V$.
\end{proof}

\section*{Problem 3}
\begin{theorem}
Let $V \in \F^n$ with that standard inner product and $A \in M_{n \times n}(\F)$. Then the following statements are true:
\proofpart{(a)}{} $\iprod{x, Ay} = \iprod{A^* x, y}$

\proofpart{(b)}{} If for some $B \in M_{n \times n}(\F)$ we have $\iprod{x, Ay} = \iprod{B x, y}$ for all $x, y \in V$, then $B = A^*$.
\end{theorem}

\subsection*{Solution}
\begin{proof}
\proofpart{(a)}{} Using the definition of our standard inner product, we have
\begin{align*}
\iprod{x, Ay} &= \sum_{i = 1}^n x_i \overline{(Ay)_i}\\
&= \sum_{i = 1}^n x_i \overline{\sum_{j = 1}^n A_{i, j}y_j}\\
&= \sum_{i = 1}^n x_i \sum_{j = 1}^n \overline{A_{i, j}y_j}\\
&= \sum_{i = 1}^n \sum_{j = 1}^n x_i \overline{A_{i, j}} \overline{y_j}\\
&= \sum_{i = 1}^n \sum_{j = 1}^n A_{j, i}^* x_i \overline{y_j}\\
&= \sum_{j = 1}^n \left( \sum_{i = 1}^n A_{j, i}^* x_i \right) \overline{y_j}\\
&= \sum_{j = 1}^n (A^*x)_j \overline{y_j}\\
&= \iprod{A^*x, y},
\end{align*}
as desired.

\proofpart{(b)}{} For any two vectors $x,y \in V$, we have
\begin{align*}
\iprod{Bx, y} &= \iprod{x, Ay}\\
&= \iprod{A^* x, y}
\end{align*}
by part (a). Thus, for any $x \in V$, we have
\begin{align*}
0 &= \iprod{Bx, Bx - A^*x} - \iprod{A^* x, Bx - A^*x}\\
&= \iprod{Bx - A^* x, Bx - A^*x}.
\end{align*}
By taking the contrapositive of property (d) of inner product, we have that 
\begin{align*}
Bx - A^*x = 0 \implies Bx = A^*x.
\end{align*}
Since this is true for any $x \in V$, we can conclude that $B = A^*$.
\end{proof}
\end{document}