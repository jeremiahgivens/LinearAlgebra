\documentclass[10pt,a4paper]{article}
\usepackage[utf8]{inputenc}
\usepackage[a4paper,%
            left=.75in,right=.75in,top=1in,bottom=1in]{geometry}
\setlength{\headsep}{0.25in}

\usepackage{amsthm}

\usepackage{graphicx}
\usepackage{pgfplots}
            
\usepackage[english]{babel}

\newtheorem{theorem}{Theorem}
\newtheorem{lemma}{Lemma}
\newtheorem{corollary}{Corollary}
\newtheorem{case}{Case}

\usepackage{amsthm}
\usepackage{lipsum}
\usepackage{tikz}

\makeatletter
\newcommand{\proofpart}[2]{%
  \par
  \addvspace{\medskipamount}%
  \noindent\emph{Part #1: #2}\par\nobreak
  \addvspace{\smallskipamount}%
  \@afterheading
}
\makeatother

\newcommand\restr[2]{{% we make the whole thing an ordinary symbol
  \left.\kern-\nulldelimiterspace % automatically resize the bar with \right
  #1 % the function
  \vphantom{\big|} % pretend it's a little taller at normal size
  \right|_{#2} % this is the delimiter
  }}

\theoremstyle{definition}
\newtheorem{definition}{Definition}
\newtheorem{remark}{Remark}

\usepackage{mathtools}
\DeclarePairedDelimiter\bra{\langle}{\rvert}
\DeclarePairedDelimiter\ket{\lvert}{\rangle}
\DeclarePairedDelimiterX\braket[2]{\langle}{\rangle}{#1 \delimsize\vert #2}

\usepackage{amsmath}
\usepackage{amsfonts}
\usepackage{amssymb}
\usepackage{fancyhdr}
\usepackage{tkz-euclide}

\DeclareMathOperator{\interior}{int}

\newcommand{\Tau}{\mathcal{T}}

\newenvironment{amatrix}[1]{%
  \left(\begin{array}{@{}*{#1}{c}|c@{}}
}{%
  \end{array}\right)
}

\usepackage{calligra}
\DeclareMathAlphabet{\mathcalligra}{T1}{calligra}{m}{n}
\DeclareFontShape{T1}{calligra}{m}{n}{<->s*[2.2]callig15}{}

\newcommand{\scripty}[1]{\ensuremath{\mathcalligra{#1}}}

\pagestyle{fancy}
\author{Jeremiah Givens}
\newcommand{\subject}{Linear Algebra}
\newcommand{\Date}{9/2/2021} 
\makeatletter
\rhead{{\small\@author}}
\lhead{{\small\subject}}
\chead{{\large Homework 1}}
\cfoot{}
\rfoot{\thepage}
\lfoot{\today}

\renewcommand{\theequation}{\arabic{equation}}

\begin{document}
\section*{Problem 1}
\begin{theorem}
\proofpart{(a)}{} The skew symmetric matrices form a subspace of $M_{n \times n} (\mathbb{R})$.
\proofpart{(b)}{} If $W$ denotes the subspace of symmetric matrices, and $W_-$ denotes the subspace of antisymmetric matrices, then $M_{n \times n} (\mathbb{R}) = W + W_-$ and $W \cap W_- = \{0\}$.
\end{theorem}

\subsection*{Solution}
\begin{proof}
\proofpart{(a)}{} Since each entry of the zero matrix is zero, we have $0_{ij} = -0_{ji}$, and can conclude that the zero matrix is skew symmetric. Now, let $w, x \in M_{n \times n} (\mathbb{R})$ be skew symmetric. Then, for $1 \leq i,j \leq n$,
\begin{align*}
(w + x)_{ji} &= w_{ji} + x_{ji} &&\text{Basic property of matrix addition}\\
&= -w_{ij} - x_{ij} &&\text{Both vectors are skew symmetric}\\
&= -(w + x)_{ij},
\end{align*}
and we have shown that $w + x$ is skew symmetric.

Finally, let $a \in \mathbb{R}$ and $v \in M_{n \times n} (\mathbb{R})$ be skew symmetric. For $1 \leq i,j \leq n$, we have
\begin{align*}
(ax)_{ji} &= a (x_{ji})\\
&= a (-x_{ij}) &&x \text{ is skew symmetric}\\
&= -(ax)_{ij},
\end{align*}
and we have proven that $ax$ is skew symmetric. With this, we have proven that the subset of skew symmetric matrices forms a subspace.

\proofpart{(b)}{} Let $v \in M_{n \times n} (\mathbb{R})$. Define $w, x \in M_{n \times n} (\mathbb{R})$ by 
\begin{align*}
w_{ij} = \frac{1}{2}(v_{ij} - v_{ji}) \text{, and } x_{ij} = \frac{1}{2}(v_{ij} + v_{ji})
\end{align*}
for $1 \leq i,j \leq n$. Then, we have 
\begin{align*}
w_{ji} &= \frac{1}{2}(v_{ji} - v_{ij}) \\
&= - \frac{1}{2}(v_{ij} - v_{ji})\\
&= - w_{ij}\text{,}
\end{align*}
and we have that $w \in W_-$. Similarly,
\begin{align*}
x_{ji} &= \frac{1}{2}(v_{ji} + v_{ij})\\
&= \frac{1}{2}(v_{ij} + v_{ji})\\
&= x_{ij},
\end{align*}
and we have shown that $x \in W$. 

From the definition of $x$ and $w$, we have 
\begin{align*}
w_{ij} + x_{ij} &= \frac{1}{2}(v_{ij} - v_{ji}) + \frac{1}{2}(v_{ij} + v_{ji})\\
&= v_{ij},
\end{align*}
and we can conclude that $v = w + x$. Since $v$ was arbitrary, we have shown any $W + W_- = V$.

Finally, we have
\begin{align*}
v \in W \cap W_- &\implies (\forall 1 \leq i, j \leq n)(v_{ij} = v_{ji} \land v_{ij} = -v_{ji})\\
&\implies (\forall 1 \leq i, j \leq n)(v_{ij} = 0)\\
&\implies v = 0.
\end{align*}
Thus, we have shown $W \cap W_- = \{0\}$, and our proof is complete.
\end{proof}

\section*{Problem 2}
\begin{theorem}
Let $T$ be an invertible linear transformation $T:V \to W$, and let $\mathcal{B} = \{v_1, v_2, ..., v_n \}$ be a basis for $V$. Then $\{T(v_1), T(v_2),..., T(v_n) \}$ is a basis for $W$.  
\end{theorem}
\subsection*{Solution}
\begin{proof}
For this proof, we will show that $\{T(v_1), T(v_2),..., T(v_n) \}$ spans $V$ and is linearly independent. Let $w \in W$. Then, since $T$ is invertible, it is surjective, which implies there exists a $v \in V$ such that $T(v) = w$. Since $\mathcal{B}$ is a basis of $V$, there exist $a_i \in \mathbb{F}$ for $1 \leq i \leq n$ such that 
\begin{align*}
v = \sum_{i = 1}^n a_iv_i.
\end{align*}
Then, using the fact that $T$ is linear, we have
\begin{align*}
w &= T(v)\\
&= T \left( \sum_{i = 1}^n a_i v_i \right)\\
&= \sum_{i = 1}^n T(a_iv_i)\\
&= \sum_{i = 1}^n a_i T(v_i),
\end{align*}
and we have shown that $\{T(v_1), T(v_2),..., T(v_n) \}$ spans $V$. 

Finally, let $a_i \in \mathbb{F}$ for $1 \leq i \leq n$, be such that 
\begin{align}
\sum_{i = 1}^n a_i T(v_i) &= 0.
\end{align}
Then, we have
\begin{align*}
0 &= \sum_{i = 1}^n a_i T(v_i)\\
&= \sum_{i = 1}^n T(a_iv_i)\\
&= T \left( \sum_{i = 1}^n a_i v_i \right).
\end{align*}
By linearity of $T$, $T(0) = 0$. Furthermore, since $T$ is injective (as a result of invertibility), we can conclude 
\begin{align*}
\sum_{i = 1}^n a_i v_i = 0.
\end{align*}
By linear independence of $\mathcal{B}$, we have $a_i = 0$ for all $i$. Thus, the only solution to (1) is the trivial one, implying that $\{T(v_1), T(v_2),..., T(v_n) \}$ is linearly independent. With this, our proof is complete.
\end{proof}

\end{document}