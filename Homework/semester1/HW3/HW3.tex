\documentclass[10pt,a4paper]{article}
\usepackage[utf8]{inputenc}
\usepackage[a4paper,%
            left=.75in,right=.75in,top=1in,bottom=1in]{geometry}
\setlength{\headsep}{0.25in}

\usepackage{amsthm}

\usepackage{graphicx}
\usepackage{pgfplots}
            
\usepackage[english]{babel}

\newtheorem{theorem}{Theorem}
\newtheorem{lemma}{Lemma}
\newtheorem{corollary}{Corollary}
\newtheorem{case}{Case}

\usepackage{amsthm}
\usepackage{lipsum}
\usepackage{tikz}

\makeatletter
\newcommand{\proofpart}[2]{%
  \par
  \addvspace{\medskipamount}%
  \noindent\emph{Part #1: #2}\par\nobreak
  \addvspace{\smallskipamount}%
  \@afterheading
}
\makeatother

\newcommand\restr[2]{{% we make the whole thing an ordinary symbol
  \left.\kern-\nulldelimiterspace % automatically resize the bar with \right
  #1 % the function
  \vphantom{\big|} % pretend it's a little taller at normal size
  \right|_{#2} % this is the delimiter
  }}

\theoremstyle{definition}
\newtheorem{definition}{Definition}
\newtheorem{remark}{Remark}

\usepackage{mathtools}
\DeclarePairedDelimiter\bra{\langle}{\rvert}
\DeclarePairedDelimiter\ket{\lvert}{\rangle}
\DeclarePairedDelimiterX\braket[2]{\langle}{\rangle}{#1 \delimsize\vert #2}

\usepackage{amsmath}
\usepackage{amsfonts}
\usepackage{amssymb}
\usepackage{fancyhdr}
\usepackage{tkz-euclide}

\DeclareMathOperator{\interior}{int}

\newcommand{\Tau}{\mathcal{T}}

\newenvironment{amatrix}[1]{%
  \left(\begin{array}{@{}*{#1}{c}|c@{}}
}{%
  \end{array}\right)
}

\usepackage{calligra}
\DeclareMathAlphabet{\mathcalligra}{T1}{calligra}{m}{n}
\DeclareFontShape{T1}{calligra}{m}{n}{<->s*[2.2]callig15}{}

\newcommand{\scripty}[1]{\ensuremath{\mathcalligra{#1}}}

\pagestyle{fancy}
\author{Jeremiah Givens}
\newcommand{\subject}{Linear Algebra}
\newcommand{\Date}{9/2/2021} 
\makeatletter
\rhead{{\small\@author}}
\lhead{{\small\subject}}
\chead{{\large Homework 3}}
\cfoot{}
\rfoot{\thepage}
\lfoot{\today}

\renewcommand{\theequation}{\arabic{equation}}

\begin{document}
\section*{Problem 1}
Determine if the following matrices aer diagonalizable over the field $\mathbb{R}$. If the matrix is diagonalizable find the eigenspaces and the matrix $Q$ which diagonalizes the matrix.
\proofpart{(a)}{}
$
\begin{bmatrix}
1 && 2\\
0 && 1
\end{bmatrix}
$
\proofpart{(b)}{}
$
\begin{bmatrix}
0 && 0 && -1\\
1 && 0 && -1\\
0 && 1 && 1
\end{bmatrix}
$
\proofpart{(b)}{}
$
\begin{bmatrix}
7 && -4 && 0\\
8 && -5 && 0\\
6 && -6 && 3
\end{bmatrix}
$

\subsection*{Solution}
For each of these matrices, we will begin by finding the eigen values. With these eigen values, we will then find each corresponding eigenspaces. If the sum of the dimensions of these eigenspaces is equal to the dimension of vector space these matrices act on, then we will construct a basis of our vector space consisting of eigenvectors, and use this to find the matrix which diagonlizes our matrix.
\proofpart{(a)}{} We have
\begin{align*}
\begin{vmatrix}
1 - \lambda && 2\\
0 && 1 - \lambda
\end{vmatrix} &= (1 - \lambda)^2\\
&= 0,
\end{align*}
which means the eigen value of this matrix is $\lambda = 1$, with a multiplicity of 2. Then, to find the eigenspace corresponding to this eigenvalue,
\begin{align*}
\begin{bmatrix}
1 - \lambda && 2\\
0 && 1 - \lambda
\end{bmatrix} &= \begin{bmatrix}
0 && 2\\
0 && 0
\end{bmatrix}.
\end{align*}
Thus, we have
\begin{align*}
E_1 = \Biggl\{ \begin{bmatrix}
s\\
0
\end{bmatrix} : s \in \mathbb{R} \Biggr\}.
\end{align*}
Since the dimension of the space spanned by all of our eigenvectors is less than the dimension of $\mathbb{R}^2$, we have shown this matrix is not diagonalizable.
\proofpart{(b)}{} Begining as before,
\begin{align*}
\begin{vmatrix}
-\lambda && 0 && -1\\
1 && -\lambda && -1\\
0 && 1 && 1 - \lambda
\end{vmatrix} &= -\lambda \begin{vmatrix}
-\lambda && -1\\
1 && 1 - \lambda \end{vmatrix} - \begin{vmatrix}
1 && -\lambda\\
0 && 1
\end{vmatrix}\\
&= -\lambda (-\lambda(1 - \lambda) + 1) - 1\\
&= -\lambda^3 + \lambda^2  - \lambda - 1.
\end{align*}
Plugging this polynomial into wolfram alpha, we see that it has only 1 real root, and two complex roots. Thus, since we are working over $\mathbb{R}$, this matrix is not diagonlizable.
\proofpart{(c)}{} Finally, 
\begin{align*}
0 &= 
\begin{vmatrix}
7 - \lambda && -4 && 0\\
8 && -5 - \lambda && 0\\
6 && -6 && 3 - \lambda
\end{vmatrix}\\ &= (3 - \lambda) \begin{vmatrix}
7 - \lambda && -4\\
8 && -5 - \lambda
\end{vmatrix}\\
&= (3 - \lambda)[(7- \lambda)(-5 - \lambda) + 32]\\
&= (3 - \lambda)[-35 - 2 \lambda + \lambda^2 + 32]\\
&= (3 - \lambda)[-3 -2 \lambda + \lambda^2]\\
&= (3 - \lambda)(\lambda + 1)(\lambda - 3)\\
&= (3 - \lambda)^2(\lambda + 1).
\end{align*}
With this, we can now find the eigen vectors corresponding to $\lambda = 3$ and $\lambda = -1$:
\begin{align*}
\begin{bmatrix}
7 - 3 && -4 && 0\\
8 && -5 - 3 && 0\\
6 && -6 && 3 - 3
\end{bmatrix} &= \begin{bmatrix}
4 && -4 && 0\\
8 && -8 && 0\\
6 && -6 && 0
\end{bmatrix}\\
&\to \begin{bmatrix}
4 && -4 && 0\\
0 && 0 && 0\\
0 && 0 && 0
\end{bmatrix} && R_3 - \frac{6}{4}R_1 \land R_2 - 2R_1\\
&\to \begin{bmatrix}
1 && -1 && 0\\
0 && 0 && 0\\
0 && 0 && 0
\end{bmatrix} && R_3 - \frac{6}{4}R_1 \land R_2 - 2R_1
\end{align*}
Thus, we have
\begin{align*}
E_3 &= \Biggl\{ \begin{bmatrix}
t\\
t\\
s
\end{bmatrix} : s,t \in \mathbb{R} \Biggr\}\\
&= \text{Span} \Biggl\{ \begin{bmatrix}
1\\
1\\
0
\end{bmatrix} , \begin{bmatrix}
0\\
0\\
1
\end{bmatrix} \Biggr\}\\
\end{align*}
which has dimension 2.

Now, for our remaining eigenvalue, we have 
\begin{align*}
\begin{bmatrix}
7 + 1 && -4 && 0\\
8 && -5 + 1 && 0\\
6 && -6 && 3 + 1
\end{bmatrix} &= \begin{bmatrix}
8 && -4 && 0\\
8 && -4 && 0\\
6 && -6 && 4
\end{bmatrix}\\
&\to \begin{bmatrix}
8 && -4 && 0\\
0 && 0 && 0\\
6 && -6 && 4
\end{bmatrix} && R_2 - R_1\\
&\to \begin{bmatrix}
8 && -4 && 0\\
6 && -6 && 4\\
0 && 0 && 0
\end{bmatrix} && \text{Row swapping}\\
&\to \begin{bmatrix}
8 && -4 && 0\\
0 && -3 && 4\\
0 && 0 && 0
\end{bmatrix} && R_2 - \frac{6}{8}R_1\\
&\to \begin{bmatrix}
2 && -1 && 0\\
0 && 1 && -\frac{4}{3}\\
0 && 0 && 0
\end{bmatrix} && \frac{1}{8}R_1 \land -\frac{1}{3}R_2\\
\end{align*}
Then,
\begin{align*}
E_{-1} = \text{Span} \Biggl\{ \begin{bmatrix}
2\\
4\\
3
\end{bmatrix} \Biggr\}.
\end{align*}
With this, we can define a basis of $\mathbb{R}^3$
\begin{align*}
\{v_1, v_2, v_3 \} = \Biggl\{ \begin{bmatrix}
2\\
4\\
3
\end{bmatrix} , \begin{bmatrix}
1\\
1\\
0
\end{bmatrix} , \begin{bmatrix}
0\\
0\\
1
\end{bmatrix}\Biggr\}.
\end{align*}
The change of basis matrix from our eigenbasis to the standard basis is then
\begin{align*}
Q = \begin{bmatrix}
2 & 1 & 0\\
4 & 1 & 0\\
3 & 0 & 1
\end{bmatrix}.
\end{align*}
To find the inverse matrix, we augment to the identity matrix and reduce to RREF:
\begin{align*}
[Q | I] &= \begin{bmatrix}
2 & 1 & 0 & 1 & 0 & 0\\
4 & 1 & 0 & 0 & 1 & 0\\
3 & 0 & 1 & 0 & 0 & 1
\end{bmatrix}\\ 
&\to \begin{bmatrix}
2 & 1 & 0 & 1 & 0 & 0\\
0 & -1 & 0 & -2 & 1 & 0\\
3 & 0 & 1 & 0 & 0 & 1
\end{bmatrix} \\
&\to \begin{bmatrix}
2 & 1 & 0 & 1 & 0 & 0\\
0 & 1 & 0 & 2 & -1 & 0\\
3 & 0 & 1 & 0 & 0 & 1
\end{bmatrix} \\
&\to \begin{bmatrix}
2 & 0 & 0 & -1 & 1 & 0\\
0 & 1 & 0 & 2 & -1 & 0\\
3 & 0 & 1 & 0 & 0 & 1
\end{bmatrix} \\
&\to \begin{bmatrix}
1 & 0 & 0 & -1/2 & 1/2 & 0\\
0 & 1 & 0 & 2 & -1 & 0\\
3 & 0 & 1 & 0 & 0 & 1
\end{bmatrix} \\
&\to \begin{bmatrix}
1 & 0 & 0 & -1/2 & 1/2 & 0\\
0 & 1 & 0 & 2 & -1 & 0\\
0 & 0 & 1 & 3/2 & -3/2 & 1
\end{bmatrix} \\
&= [I | Q^{-1}].
\end{align*}
Therefore, we have 
\begin{align*}
Q^{-1} &= \begin{bmatrix}
-1/2 & 1/2 & 0\\
2 & -1 & 0\\
3/2 & -3/2 & 1
\end{bmatrix},
\end{align*}
and, if we let our matrix be denoted $A$, we can see
\begin{align*}
Q^{-1} A Q &= 
\begin{bmatrix}
-1/2 & 1/2 & 0\\
2 & -1 & 0\\
3/2 & -3/2 & 1
\end{bmatrix}\begin{bmatrix}
7 && -4 && 0\\
8 && -5 && 0\\
6 && -6 && 3
\end{bmatrix} \begin{bmatrix}
2 & 1 & 0\\
4 & 1 & 0\\
3 & 0 & 1
\end{bmatrix}   \\
&= \begin{bmatrix}
-1 & 0 & 0\\
0 & 3 & 0\\
0 & 0 & 3\\
\end{bmatrix}.
\end{align*}

\section*{Problem 2}
\begin{theorem}
\proofpart{(a)}{} Not every invertible matrix is diagonlizable.
\proofpart{(b)}{} Not every diagonlizable matrix is invertible.
\end{theorem}

\subsection*{Solution}
\begin{proof}
\proofpart{(a)}{} Consider a matrix representing a counter clockwise rotation in $\mathbb{R}^2$ by $0 < \theta < 2 \pi$ radians. This matrix is invertible (inverse matrix rotates by $\theta$ radians clockwise). Now, since the only vector that remains pointing in the same direction after being rotated by $\theta$ radians is the zero vector, we can conclude this matrix has not eigenvectors, and is therefore not diagonlizable.

\proofpart{(b)}{} Let $A$ be a matrix in $\mathbb{R}^2$ with eigen values $0$ and $1$. These eigen values are unique, thus the correspond to two eigen vectors that span $\mathbb{R}^2$. However, this means that $A$ has a nonzero vector in it's nullspace, which implies it is not injective (since $A$ is linear), and therefore non-invertible.
\end{proof}

\section*{Problem 3}
In this question, you may only assume that $|AB| = |A||B|$ for $n \times n$ matrices $A$ and $B$ with entries from a field $F$.
\begin{theorem}
\proofpart{(a)}{} If $Q$ is an $n \times n$ invertible matrix, then $|Q^{-1}| = |Q|^{-1}$.
\proofpart{(b)}{} Similar matrices have the same determinant.
\proofpart{(c)}{} Similar matrices have the same characteristic polynomial.
\end{theorem}

\subsection*{Solution}
\begin{proof}
\proofpart{(a)}{} We have
\begin{align*}
1 &= |I| \\
&= |Q Q^{-1}|\\
&= |Q| |Q^{-1}|\\
|Q|^{-1} &= |Q^{-1}|,
\end{align*}
as desired.
\proofpart{(b)}{} Let $A$ and $Q$ be $n \times n$ matrices with $Q$ having inverse $Q^{-1}$. Then, we have
\begin{align*}
|Q A Q^{-1}| &= |Q| |A| |Q^{-1}|\\
&= |A| &&\text{By Part (a).}
\end{align*}
Thus, similar matrices have the same determinant.
\proofpart{(c)}{} Defining $A$ and $Q$ as before, and let 
\begin{align*}
B = Q A Q^{-1}.
\end{align*}
Then, we have
\begin{align*}
|B - \lambda I| &= |Q A Q^{-1} - \lambda I|\\
&= |Q^{-1}(Q A Q^{-1} - \lambda I) Q| && \text{By Part (b)}\\
&= |Q^{-1}Q A Q^{-1}Q - Q^{-1}\lambda I Q| \\
&= |IAI - \lambda I|\\
&= |A - \lambda I|,
\end{align*}
and we have shown that similar matrices have the same characteristic polynomial.
\end{proof}
\end{document}