\documentclass[10pt,a4paper]{article}
\usepackage[utf8]{inputenc}
\usepackage[a4paper,%
            left=.75in,right=.75in,top=1in,bottom=1in]{geometry}
\setlength{\headsep}{0.25in}

\usepackage{amsthm}

\usepackage{graphicx}
\usepackage{pgfplots}
            
\usepackage[english]{babel}

\newtheorem{theorem}{Theorem}
\newtheorem{lemma}{Lemma}
\newtheorem{corollary}{Corollary}
\newtheorem{case}{Case}

\usepackage{amsthm}
\usepackage{lipsum}
\usepackage{tikz}

\makeatletter
\newcommand{\proofpart}[2]{%
  \par
  \addvspace{\medskipamount}%
  \noindent\emph{Part #1: #2}\par\nobreak
  \addvspace{\smallskipamount}%
  \@afterheading
}
\makeatother

\newcommand\restr[2]{{% we make the whole thing an ordinary symbol
  \left.\kern-\nulldelimiterspace % automatically resize the bar with \right
  #1 % the function
  \vphantom{\big|} % pretend it's a little taller at normal size
  \right|_{#2} % this is the delimiter
  }}

\theoremstyle{definition}
\newtheorem{definition}{Definition}
\newtheorem{remark}{Remark}

\usepackage{mathtools}
\DeclarePairedDelimiter\bra{\langle}{\rvert}
\DeclarePairedDelimiter\ket{\lvert}{\rangle}
\DeclarePairedDelimiterX\braket[2]{\langle}{\rangle}{#1 \delimsize\vert #2}

\usepackage{amsmath}
\usepackage{amsfonts}
\usepackage{amssymb}
\usepackage{fancyhdr}
\usepackage{tkz-euclide}

\DeclareMathOperator{\interior}{int}

\newcommand{\Tau}{\mathcal{T}}

\newenvironment{amatrix}[1]{%
  \left(\begin{array}{@{}*{#1}{c}|c@{}}
}{%
  \end{array}\right)
}

\usepackage{calligra}
\DeclareMathAlphabet{\mathcalligra}{T1}{calligra}{m}{n}
\DeclareFontShape{T1}{calligra}{m}{n}{<->s*[2.2]callig15}{}

\newcommand{\scripty}[1]{\ensuremath{\mathcalligra{#1}}}

\pagestyle{fancy}
\author{Jeremiah Givens}
\newcommand{\subject}{Linear Algebra}
\newcommand{\Date}{9/2/2021} 
\makeatletter
\rhead{{\small\@author}}
\lhead{{\small\subject}}
\chead{{\large Exam 1 Corrections}}
\cfoot{}
\rfoot{\thepage}
\lfoot{\today}

\renewcommand{\theequation}{\arabic{equation}}

\begin{document}
\section*{Problem 5}
\begin{theorem}
Let $T$ be a linear operator on a finite dimensional vector space $V$, let $v \in V$ be nonzero, and let $W$ be the $T$-cyclic subspace of $V$ of dimension $k$ generated by $v$
\begin{align*}
W = \text{Span} \{v, T(v),\cdots, T^{k-1}(v) \}.
\end{align*} 
If $S$ is a $T$-invariant subspace of $V$, then $W \subseteq S$.
\end{theorem}

\subsection*{Solution}
\begin{proof}
Since $v \in S$, the definition of an invariant subspace tells us that $T(v) \in S$. Now suppose that for some natural $n \geq 0$ that $T^n(v) \in S$. Then, by invariance, we have $T(T^n(v)) = T^{n + 1}(v) \in S$. Thus, for all nonnegative integers $n$, we have
\begin{align*}
T^n(v) \in S.
\end{align*}
Since $S$ is a subspace, it also contains the span of these vectors, and we can conclude that $W \subseteq S$.
\end{proof}
\end{document}