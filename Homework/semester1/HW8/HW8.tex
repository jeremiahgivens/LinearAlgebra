\documentclass[10pt,a4paper]{article}
\usepackage[utf8]{inputenc}
\usepackage[a4paper,%
            left=.75in,right=.75in,top=1in,bottom=1in]{geometry}
\setlength{\headsep}{0.25in}

\usepackage{amsthm}

\usepackage{graphicx}
\usepackage{pgfplots}
            
\usepackage[english]{babel}

\newtheorem{theorem}{Theorem}
\newtheorem{lemma}{Lemma}
\newtheorem{corollary}{Corollary}
\newtheorem{case}{Case}

\usepackage{amsthm}
\usepackage{lipsum}
\usepackage{tikz}

\makeatletter
\newcommand{\proofpart}[2]{%
  \par
  \addvspace{\medskipamount}%
  \noindent\emph{Part #1: #2}\par\nobreak
  \addvspace{\smallskipamount}%
  \@afterheading
}
\makeatother

\newcommand\restr[2]{{% we make the whole thing an ordinary symbol
  \left.\kern-\nulldelimiterspace % automatically resize the bar with \right
  #1 % the function
  \vphantom{\big|} % pretend it's a little taller at normal size
  \right|_{#2} % this is the delimiter
  }}

\theoremstyle{definition}
\newtheorem{definition}{Definition}
\newtheorem{remark}{Remark}

\usepackage{mathtools}
\DeclarePairedDelimiter\bra{\langle}{\rvert}
\DeclarePairedDelimiter\ket{\lvert}{\rangle}
\DeclarePairedDelimiterX\braket[2]{\langle}{\rangle}{#1 \delimsize\vert #2}

\usepackage{amsmath}
\usepackage{amsfonts}
\usepackage{amssymb}
\usepackage{fancyhdr}
\usepackage{tkz-euclide}

\DeclareMathOperator{\interior}{int}

\newcommand{\Tau}{\mathcal{T}}
\newcommand{\F}{\mathbb{F}}
\newcommand{\R}{\mathbb{R}}
\newcommand{\C}{\mathbb{C}}
\newcommand{\B}{\mathcal{B}}
\DeclarePairedDelimiterX{\iprod}[1]{\langle}{\rangle}{#1}

\newenvironment{amatrix}[1]{%
  \left(\begin{array}{@{}*{#1}{c}|c@{}}
}{%
  \end{array}\right)
}

\usepackage{calligra}
\DeclareMathAlphabet{\mathcalligra}{T1}{calligra}{m}{n}
\DeclareFontShape{T1}{calligra}{m}{n}{<->s*[2.2]callig15}{}

\newcommand{\scripty}[1]{\ensuremath{\mathcalligra{#1}}}

\pagestyle{fancy}
\author{Jeremiah Givens}
\newcommand{\subject}{Linear Algebra}
\newcommand{\Date}{9/2/2021} 
\makeatletter
\rhead{{\small\@author}}
\lhead{{\small\subject}}
\chead{{\large Homework 8}}
\cfoot{}
\rfoot{\thepage}
\lfoot{\today}

\renewcommand{\theequation}{\arabic{equation}}

\begin{document}
\section*{Problem 1}
Find the singular value decomposition of the matrix
\begin{align*}
A = \begin{bmatrix}
1 & 1\\
0 & 1\\
1 & 0\\
1 & 1
\end{bmatrix}
\end{align*}

\subsection*{Solution}
We start by computing $A^*A$:
\begin{align*}
A^*A &= \begin{bmatrix}
1 & 0 & 1 & 1\\
1 & 1 & 0 & 1
\end{bmatrix} \begin{bmatrix}
1 & 1\\
0 & 1\\
1 & 0\\
1 & 1
\end{bmatrix}\\
&= \begin{bmatrix}
3 & 2\\
2 & 3
\end{bmatrix}.
\end{align*}
Now, we find the eigenvalues of this matrix:
\begin{align*}
\begin{vmatrix}
3 - \lambda & 2\\
2 & 3 - \lambda
\end{vmatrix} &= (3 - \lambda)^2 - 4\\
&= \lambda^2 - 6\lambda + 5\\
&= (\lambda - 5)(\lambda - 1).
\end{align*}
Thus, we have $\lambda_1 = 5$ and $\lambda_2 = 1$. Now we find the corresponding eigenvectors to these eigenvalues. We have
\begin{align*}
\begin{bmatrix}
3 - \lambda_1 & 2\\
2 & 3 - \lambda_1
\end{bmatrix} &= \begin{bmatrix}
3 - 5 & 2\\
2 & 3 - 5
\end{bmatrix}\\
&= \begin{bmatrix}
-2 & 2\\
2 & -2
\end{bmatrix}\\
&\to \begin{bmatrix}
2 & -2\\
2 & -2
\end{bmatrix} && -R_1\\
&\to \begin{bmatrix}
2 & -2\\
0 & 0
\end{bmatrix} && R_2 - R_1\\
&\to \begin{bmatrix}
1 & -1\\
0 & 0
\end{bmatrix} && R_2 - R_1,
\end{align*}
and we can see that
\begin{align*}
v_1 = \frac{1}{\sqrt{2}} \begin{bmatrix}
1\\
1
\end{bmatrix}
\end{align*}
is a normalized eigenvector corresponding to eigenvalue $\lambda_1$. Doing the same for our remaining eigenvalue, we have
\begin{align*}
\begin{bmatrix}
3 - \lambda_2 & 2\\
2 & 3 - \lambda_2
\end{bmatrix} &= \begin{bmatrix}
3 - 1 & 2\\
2 & 3 - 1
\end{bmatrix}\\
&= \begin{bmatrix}
2 & 2\\
2 & 2
\end{bmatrix}\\
&\to \begin{bmatrix}
1 & 1\\
0 & 0
\end{bmatrix}
\end{align*}
and we can see that 
\begin{align*}
v_2 &= \frac{1}{\sqrt{2}} \begin{bmatrix}
1\\
-1
\end{bmatrix}
\end{align*}
is a normalized eigenvector corresponding to eigenvalue $\lambda_2$.

Now, the singular values of $A$ are $\sigma_1 = \sqrt{\lambda_1} = \sqrt{5}$ and $\sigma_2 = \sqrt{\lambda_2} = 1$, and we can compute the column vectors of $u$:
\begin{align*}
u_1 &= \frac{1}{\sigma_1} A v_1\\
&= \frac{1}{5 \sqrt{2}} \begin{bmatrix}
1 & 1\\
0 & 1\\
1 & 0\\
1 & 1
\end{bmatrix} \begin{bmatrix}
1\\
1
\end{bmatrix}\\
&= \frac{1}{5 \sqrt{2}} \begin{bmatrix}
2\\
1\\
1\\
2
\end{bmatrix},
\end{align*}
and 
\begin{align*}
u_2 &= \frac{1}{\sigma_2} A v_2\\
&= \frac{1}{\sqrt{2}} \begin{bmatrix}
1 & 1\\
0 & 1\\
1 & 0\\
1 & 1
\end{bmatrix} \begin{bmatrix}
1\\
-1
\end{bmatrix}\\
&= \frac{1}{\sqrt{2}} \begin{bmatrix}
0\\
-1\\
1\\
0
\end{bmatrix}.
\end{align*}
Thus, we have
\begin{align*}
A &= U \Sigma V^*\\
&= \begin{bmatrix}
\frac{2}{5 \sqrt{2}} & 0\\
\frac{1}{5 \sqrt{2}} & -\frac{1}{\sqrt{2}}\\
\frac{1}{5 \sqrt{2}} & \frac{1}{\sqrt{2}}\\
\frac{2}{5 \sqrt{2}} & 0
\end{bmatrix} \begin{bmatrix}
\sqrt{5} & 0\\
0 & 1
\end{bmatrix} \begin{bmatrix}
\frac{1}{\sqrt{2}} & \frac{1}{\sqrt{2}}\\
\frac{1}{\sqrt{2}} & -\frac{1}{\sqrt{2}}
\end{bmatrix}.
\end{align*}

\section*{Problem 2}
If $A = (a_ij)$ is an $m\times n$ real matrix, recall that the Frobenius norm of $A$ is
\begin{align*}
||A||_F = \sqrt{\text{Tr}(A^*A)}.
\end{align*}
Show that $||A||_F = \sum \sigma_i^2$, the sum of the squares of the singular values of $A$.

\subsection*{Solution}
\begin{proof}
Suppose $A$ has $k$ singular values. Define $\sigma_i = 0$ for $k \leq i \leq n$. As we saw in the proof of the singular value theorem, $A^* A$ is diagonlizable and has eigenvalues $\sigma_i^2$. Thus, there exists an invertible matrix $Q$ such that 
\begin{align*}
Q A^* A Q^{-1} &= \begin{bmatrix}
    \sigma_1^2 & & 0 \\
    & \ddots & \\
    0& & \sigma_n^2
  \end{bmatrix}.
\end{align*}
Finally, we have 
\begin{align*}
||A||^2_F &= \text{Tr}(A^*A)\\
&= \text{Tr}(Q A^* A Q^{-1}) && \text{Proved in part (c) of problem 3 of HW4}\\
&= \text{Tr} \left(\begin{bmatrix}
    \sigma_1^2 & & 0 \\
    & \ddots & \\
    0& & \sigma_n^2
  \end{bmatrix} \right)\\
&= \sum_{i=1}^n \sigma_i^2\\
&= \sum_{i=1}^k \sigma_i^2,
\end{align*}
as desired.
\end{proof}

\section*{Problem 3}
Find the polar decomposition of the matrix 
\begin{align*}
A = \begin{bmatrix}
1 & 1\\
2 & -2
\end{bmatrix}.
\end{align*}

\subsection*{Solution}
Starting as usual, 
\begin{align*}
A^* A &= \begin{bmatrix}
1 & 2\\
1 & -2
\end{bmatrix} \begin{bmatrix}
1 & 1\\
2 & -2
\end{bmatrix}\\
&= \begin{bmatrix}
5 & -3\\
-3 & 5
\end{bmatrix}.
\end{align*}
Now we find the eigenvalues:
\begin{align*}
\begin{vmatrix}
5 - \lambda & -3\\
-3 & 5 - \lambda
\end{vmatrix} &= (5 - \lambda)^2 - 9\\
&= \lambda^2 -10 \lambda + 16\\
&= (\lambda - 2)(\lambda - 8).
\end{align*}
Thus, $\sigma_1^2 = 8$ and $\sigma_2^2 = 2$. Finding the corresponding eigenvectors:
\begin{align*}
\begin{bmatrix}
5 - 8 & -3\\
-3 & 5 - 8
\end{bmatrix} &= \begin{bmatrix}
-3 & -3\\
-3 & -3
\end{bmatrix},
\end{align*}
and we can see we have a normalized eigenvector of 
\begin{align*}
v_1 = \begin{bmatrix}
\frac{1}{\sqrt{2}}\\
\frac{-1}{\sqrt{2}}
\end{bmatrix}.
\end{align*}
The other one we can easily get by looking:
\begin{align*}
v_2 = \begin{bmatrix}
\frac{1}{\sqrt{2}}\\
\frac{1}{\sqrt{2}}
\end{bmatrix}.
\end{align*}
We will now compute our $u_i$:
\begin{align*}
u_1 &= \frac{1}{\sigma_1} A v_1\\
&= \frac{1}{2\sqrt{2}} \begin{bmatrix}
1 & 1\\
2 & -2
\end{bmatrix} \begin{bmatrix}
\frac{1}{\sqrt{2}}\\
\frac{-1}{\sqrt{2}}
\end{bmatrix}\\
&= \begin{bmatrix}
0\\
1
\end{bmatrix}.
\end{align*}
We know $u_2$ is orthonormal to this, so we immediately have
\begin{align*}
u_2 = \begin{bmatrix}
1\\
0
\end{bmatrix}.
\end{align*}
Following the construction used in the proof of the polar decomposition theorem, we have
\begin{align*}
W &= UV^*\\
&= V^*\\
&= \begin{bmatrix}
\frac{1}{\sqrt{2}} & \frac{-1}{\sqrt{2}}\\
\frac{1}{\sqrt{2}} & \frac{1}{\sqrt{2}}
\end{bmatrix},
\end{align*}
and
\begin{align*}
P &= V \Sigma V^*\\
&= \begin{bmatrix}
\frac{1}{\sqrt{2}} & \frac{1}{\sqrt{2}}\\
\frac{-1}{\sqrt{2}} & \frac{1}{\sqrt{2}}
\end{bmatrix} \begin{bmatrix}
2\sqrt{2} & 0\\
0 & \sqrt{2}
\end{bmatrix} \begin{bmatrix}
\frac{1}{\sqrt{2}} & \frac{-1}{\sqrt{2}}\\
\frac{1}{\sqrt{2}} & \frac{1}{\sqrt{2}}
\end{bmatrix}\\
&= \begin{bmatrix}
\frac{1}{\sqrt{2}} & \frac{1}{\sqrt{2}}\\
\frac{-1}{\sqrt{2}} & \frac{1}{\sqrt{2}}
\end{bmatrix} \begin{bmatrix}
2 & -2\\
1 & 1
\end{bmatrix}\\
&= \frac{1}{\sqrt{2}} \begin{bmatrix}
3 & -1\\
-1 & 3
\end{bmatrix}.
\end{align*}
Thus, we have found the unitary matrix $W$ and positive semidefinite matrix $P$ such that $A = WP$, which means we have found a polar decomposition of $A$.

\section*{Problem 4}
Use the pseudoinverse to find the least-square solution to the following system of equations:
\begin{align*}
x + y + z = 5 \text{, and } 2x - y + z = 2.
\end{align*}

\subsection*{Solution}
The matrix for this system of equations is
\begin{align*}
A &= \begin{bmatrix}
1 & 1 & 1\\
2 & -1 & 1
\end{bmatrix}.
\end{align*}
To find the psuedoinverse, we will first find the matrices used in the singular value decomposition of $A$. We have 
\begin{align*}
A^* A &= \begin{bmatrix}
1 & 2\\
1 & -1\\
1 & 1
\end{bmatrix} \begin{bmatrix}
1 & 1 & 1\\
2 & -1 & 1
\end{bmatrix}\\
&= \begin{bmatrix}
5 & -1 & 3\\
-1 & 2 & 0\\
3 & 0 & 2
\end{bmatrix}.
\end{align*}
Now we find the eigenvalues:
\begin{align*}
\begin{vmatrix}
5 - \lambda & -1 & 3\\
-1 & 2 - \lambda & 0\\
3 & 0 & 2 - \lambda
\end{vmatrix} &= 3 \begin{vmatrix}
-1 & 2 - \lambda\\
3 & 0
\end{vmatrix} + (2-\lambda) \begin{vmatrix}
5 - \lambda & -1\\
-1 & 2 - \lambda
\end{vmatrix}\\
&= (2- \lambda) [-9 + (5-\lambda)(2 - \lambda) - 1]\\
&= (2- \lambda)[-10 + 10 - 7\lambda + \lambda^2]\\
&= \lambda (2- \lambda)(\lambda - 7).
\end{align*}
Thus, $\sigma_1^2 = 7$, and $\sigma_2^2 = 2$. Now we find the orthonormal basis (by finding our eigenvectors):
\begin{align*}
\begin{bmatrix}
5 - 7 & -1 & 3\\
-1 & 2 - 7 & 0\\
3 & 0 & 2 - 7
\end{bmatrix} &= \begin{bmatrix}
5 - 7 & -1 & 3\\
-1 & 2 - 7 & 0\\
3 & 0 & 2 - 7
\end{bmatrix}\\
&= \begin{bmatrix}
-2 & -1 & 3\\
-1 & -5 & 0\\
3 & 0 & -5
\end{bmatrix}\\
&\to \begin{bmatrix}
1 & 14 & 3\\
-1 & -5 & 0\\
3 & 0 & -5
\end{bmatrix} && R_1 - 3 R_2\\
&\to \begin{bmatrix}
1 & 14 & 3\\
0 & 9 & 3\\
0 & -42 & -14
\end{bmatrix} && R_3 - 3 R_1\\
&\to \begin{bmatrix}
1 & 14 & 3\\
0 & 1 & \frac{1}{3}\\
0 & -42 & -14
\end{bmatrix} && \frac{1}{9}R_2\\
&\to \begin{bmatrix}
1 & 0 & -\frac{5}{3}\\
0 & 1 & \frac{1}{3}\\
0 & 0 & 0
\end{bmatrix} && R_3 + 42 R_2.
\end{align*}
Thus, the vector
\begin{align*}
x = \begin{bmatrix}
5\\
-1\\
3
\end{bmatrix}
\end{align*}
is an eigenvector with eigenvalue $7$. Normalizing, we have
\begin{align*}
v_1 =  \begin{bmatrix}
\frac{5}{\sqrt{35}}\\
\frac{-1}{\sqrt{35}}\\
\frac{3}{\sqrt{35}}
\end{bmatrix}.
\end{align*}
Doing the same for our next eigenvalue:
\begin{align*}
\begin{bmatrix}
5 - 2 & -1 & 3\\
-1 & 2 - 2 & 0\\
3 & 0 & 2 - 2
\end{bmatrix} &= \begin{bmatrix}
-3 & -1 & 3\\
-1 & 0 & 0\\
3 & 0 & 0
\end{bmatrix}.
\end{align*}
Thus, the vector 
\begin{align*}
y = \begin{bmatrix}
0\\
3\\
1
\end{bmatrix}
\end{align*}
is an eigenvector with corresponding eigenvalue 2. Normalizing, we have 
\begin{align*}
v_2 = \begin{bmatrix}
0\\
\frac{3}{\sqrt{10}}\\
\frac{1}{\sqrt{10}}
\end{bmatrix}.
\end{align*}
To find our last eigenvector, we will use the cross product, as we have done enough row reduction for a lifetime:
\begin{align*}
z &= \begin{vmatrix}
i & j & k\\
5 & -1 & 3\\
0 & 3 & 1
\end{vmatrix}\\
&= i (-1 - 9) - 5 (j - 3k)\\
&= -10i -5j + 15k.
\end{align*}
Normalizing, we have
\begin{align*}
v_3 = \begin{bmatrix}
\frac{-2}{\sqrt{14}}\\
\frac{-1}{\sqrt{14}}\\
\frac{3}{\sqrt{14}}
\end{bmatrix}.
\end{align*}
Now we find our $u_i$:
\begin{align*}
u_1 &= \frac{1}{\sigma_1} A v_1\\
&= \frac{1}{\sqrt{7}} \begin{bmatrix}
1 & 1 & 1\\
2 & -1 & 1
\end{bmatrix} \begin{bmatrix}
\frac{5}{\sqrt{35}}\\
\frac{-1}{\sqrt{35}}\\
\frac{3}{\sqrt{35}}
\end{bmatrix}\\
&= \frac{1}{7 \sqrt{5}} \begin{bmatrix}
1 & 1 & 1\\
2 & -1 & 1
\end{bmatrix} \begin{bmatrix}
5\\
-1\\
3
\end{bmatrix}\\
&= \frac{1}{7 \sqrt{5}} \begin{bmatrix}
7\\
14
\end{bmatrix}\\
&= \begin{bmatrix}
\frac{1}{\sqrt{5}}\\
\frac{2}{\sqrt{5}}
\end{bmatrix},
\end{align*}
and 
\begin{align*}
u_2 &= \frac{1}{2\sqrt{5}} \begin{bmatrix}
1 & 1 & 1\\
2 & -1 & 1
\end{bmatrix} \begin{bmatrix}
0\\
3\\
1
\end{bmatrix}\\
&= \frac{1}{\sqrt{5}} \begin{bmatrix}
2\\
-1
\end{bmatrix}\\
&= \begin{bmatrix}
\frac{2}{\sqrt{5}}\\
\frac{-1}{\sqrt{5}}
\end{bmatrix}.
\end{align*}
Finally, we have
\begin{align*}
A^\dagger &= V \Sigma^\dagger U^*\\
&= \begin{bmatrix}
\frac{5}{\sqrt{35}} & 0 & \frac{-2}{\sqrt{14}}\\
\frac{-1}{\sqrt{35}} & \frac{3}{\sqrt{10}} & \frac{-1}{\sqrt{14}}\\
\frac{3}{\sqrt{35}} &  \frac{1}{\sqrt{10}} & \frac{3}{\sqrt{14}}
\end{bmatrix} \begin{bmatrix}
\frac{1}{\sqrt{7}} & 0\\
0 & \frac{1}{\sqrt{2}}\\
0 & 0.
\end{bmatrix} \begin{bmatrix}
\frac{1}{\sqrt{5}} & \frac{2}{\sqrt{5}}\\
\frac{2}{\sqrt{5}} & \frac{-1}{\sqrt{5}}
\end{bmatrix}.
\end{align*}
Thus, (assuming we somehow did not make a book keeping mistake) our least squares solution is given by
\begin{align*}
\begin{bmatrix}
\frac{5}{\sqrt{35}} & 0 & \frac{-2}{\sqrt{14}}\\
\frac{-1}{\sqrt{35}} & \frac{3}{\sqrt{10}} & \frac{-1}{\sqrt{14}}\\
\frac{3}{\sqrt{35}} &  \frac{1}{\sqrt{10}} & \frac{3}{\sqrt{14}}
\end{bmatrix} \begin{bmatrix}
\frac{1}{\sqrt{7}} & 0\\
0 & \frac{1}{\sqrt{2}}\\
0 & 0.
\end{bmatrix} \begin{bmatrix}
\frac{1}{\sqrt{5}} & \frac{2}{\sqrt{5}}\\
\frac{2}{\sqrt{5}} & \frac{-1}{\sqrt{5}}
\end{bmatrix} \begin{bmatrix}
5\\
2
\end{bmatrix} &= \frac{1}{7} \begin{bmatrix}
9\\
15\\
11
\end{bmatrix}.
\end{align*}
As a test, we can plug it in, and see how the result looks:
\begin{align*}
\frac{1}{7} \begin{bmatrix}
1 & 1 & 1\\
2 & -1 & 1
\end{bmatrix} \begin{bmatrix}
9\\
15\\
11
\end{bmatrix} &= \begin{bmatrix}
5\\
2
\end{bmatrix}.
\end{align*}
We have a solution, which is a very good sign we have done this correctly. *Breathes sigh of relief that this homework is finally over.*
\end{document}
