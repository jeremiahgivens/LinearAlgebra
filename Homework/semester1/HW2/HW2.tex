\documentclass[10pt,a4paper]{article}
\usepackage[utf8]{inputenc}
\usepackage[a4paper,%
            left=.75in,right=.75in,top=1in,bottom=1in]{geometry}
\setlength{\headsep}{0.25in}

\usepackage{amsthm}

\usepackage{graphicx}
\usepackage{pgfplots}
            
\usepackage[english]{babel}

\newtheorem{theorem}{Theorem}
\newtheorem{lemma}{Lemma}
\newtheorem{corollary}{Corollary}
\newtheorem{case}{Case}

\usepackage{amsthm}
\usepackage{lipsum}
\usepackage{tikz}

\makeatletter
\newcommand{\proofpart}[2]{%
  \par
  \addvspace{\medskipamount}%
  \noindent\emph{Part #1: #2}\par\nobreak
  \addvspace{\smallskipamount}%
  \@afterheading
}
\makeatother

\newcommand\restr[2]{{% we make the whole thing an ordinary symbol
  \left.\kern-\nulldelimiterspace % automatically resize the bar with \right
  #1 % the function
  \vphantom{\big|} % pretend it's a little taller at normal size
  \right|_{#2} % this is the delimiter
  }}

\theoremstyle{definition}
\newtheorem{definition}{Definition}
\newtheorem{remark}{Remark}

\usepackage{mathtools}
\DeclarePairedDelimiter\bra{\langle}{\rvert}
\DeclarePairedDelimiter\ket{\lvert}{\rangle}
\DeclarePairedDelimiterX\braket[2]{\langle}{\rangle}{#1 \delimsize\vert #2}

\usepackage{amsmath}
\usepackage{amsfonts}
\usepackage{amssymb}
\usepackage{fancyhdr}
\usepackage{tkz-euclide}

\DeclareMathOperator{\interior}{int}

\newcommand{\Tau}{\mathcal{T}}

\newenvironment{amatrix}[1]{%
  \left(\begin{array}{@{}*{#1}{c}|c@{}}
}{%
  \end{array}\right)
}

\usepackage{calligra}
\DeclareMathAlphabet{\mathcalligra}{T1}{calligra}{m}{n}
\DeclareFontShape{T1}{calligra}{m}{n}{<->s*[2.2]callig15}{}

\newcommand{\scripty}[1]{\ensuremath{\mathcalligra{#1}}}

\pagestyle{fancy}
\author{Jeremiah Givens}
\newcommand{\subject}{Linear Algebra}
\newcommand{\Date}{9/2/2021} 
\makeatletter
\rhead{{\small\@author}}
\lhead{{\small\subject}}
\chead{{\large Homework 2}}
\cfoot{}
\rfoot{\thepage}
\lfoot{\today}

\renewcommand{\theequation}{\arabic{equation}}

\begin{document}
\section*{Problem 1}
\begin{theorem}
Let $B$ be an $(m + 1) \times (n + 1)$ matrix, where $B_{11} = 1$, and $B_{1j} = B_{i1} = 0$ for $2 \leq j \leq n + 1$ and $2 \leq i \leq m + 1$. Define $B'$ to be the $m \times n$ submatrix of $B$ obtained by removing the first column and the first row of $B$. Then, if $\text{rank}(B) = r$, then $\text{rank}(B') = r - 1$.
\end{theorem}

\subsection*{Solution}
\begin{proof}
For this proof, we will use the fact that rank of a matrix is equal to the dimension of its column space. Since the first element of $B_{\cdot 1}$ is non zero, and the first elements of each vector in the set of remaining column vectors $\{B_{\cdot 2}, \dots, B_{\cdot (n + 1)} \}$ are zero, it follows that 
\begin{align*}
B_{\cdot 1} \not \in \text{Span}\left( \{B_{\cdot 2}, \dots, B_{\cdot (n + 1)} \} \right).
\end{align*}
Thus, we can conclude that 
\begin{align*}
\text{dim} \left( \text{Span}\left( \{B_{\cdot 2}, \dots, B_{\cdot (n + 1)} \} \right) \right) &= \text{dim} \left( \text{Span}\left( \{B_{\cdot 1}, \dots, B_{\cdot (n + 1)} \} \right) \right) - 1\\
&= r - 1.
\end{align*}
Now, define $\{v_1, ..., v_{r - 1} \}$ to be the resulting set of reducing  $\{B_{\cdot 2}, \dots, B_{\cdot (n + 1)} \}$ to a basis of $\text{Span}\left( \{B_{\cdot 2}, \dots, B_{\cdot (n + 1)} \} \right)$. Now, define a new set of vectors $\{v'_1, ..., v'_{r - 1} \} \subseteq \{ B'_{\cdot 1}, \dots, B'_{\cdot n} \}$ such that for all $1 \leq i \leq r - 1$, $v'_i$ is the vector obtained by removing the first element of $v_i$. All that remains is to that $\{v'_1, ..., v'_{r - 1} \}$ is a basis of $\text{Span}\left( B'_{\cdot 1}, \dots, B'_{\cdot n} \} \right)$.

To see that $\{v'_1, ..., v'_{r - 1} \}$ is linearly independent, let $\{a_1, \dots, a_{r-1} \} \subseteq \mathbb{F}$ be such that 
\begin{align*}
\sum_{i = 1}^{r-1} a_i v'_i = 0.
\end{align*}
Then, since the first element of each vector in $\{v_1, ..., v_{r - 1} \}$ is zero, we have
\begin{align*}
\sum_{i = 1}^{r-1} a_i v_i = 0.
\end{align*}
By linear independence of $\{v_1, ..., v_{r - 1} \}$, we can conclude that $a_i = 0$ for each $i$, and we have shown that $\{v'_1, ..., v'_{r - 1} \}$ is linearly independent.

To end the proof, we must now show that 
\begin{align*}
\text{Span}\left( \{v_1, ..., v_{r - 1} \} \right) &= \text{Span}\left( \{B'_{\cdot 1}, \dots, B'_{\cdot n} \} \right).
\end{align*}
One side of this equality follows directly from the fact that $\{v'_1, ..., v'_{r - 1} \} \subseteq \{ B'_{\cdot 1}, \dots, B'_{\cdot n} \}$. For the other direction, let $v' \in  \text{Span}\left( \{B'_{\cdot 1}, \dots, B'_{\cdot n} \} \right)$. Define $v \in \text{Span}\left( \{B_{\cdot 2}, \dots, B_{\cdot (n + 1)} \} \right)$ by adding a zero to the top of $v'$. Then, there exist scalars $\{a_1, \dots, a_{r-1} \} \subseteq \mathbb{F}$ such that 
\begin{align*}
\sum_{i = 1}^{r-1} a_i v_i = v.
\end{align*}
Since vector addition is defined element wise, it follows immediately that 
\begin{align*}
\sum_{i = 1}^{r-1} a_i v'_i = v',
\end{align*}
and our proof is complete.
\end{proof}

\end{document}