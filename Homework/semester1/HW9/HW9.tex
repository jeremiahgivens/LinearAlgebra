\documentclass[10pt,a4paper]{article}
\usepackage[utf8]{inputenc}
\usepackage[a4paper,%
            left=.75in,right=.75in,top=1in,bottom=1in]{geometry}
\setlength{\headsep}{0.25in}

\usepackage{amsthm}

\usepackage{graphicx}
\usepackage{pgfplots}
            
\usepackage[english]{babel}

\newtheorem{theorem}{Theorem}
\newtheorem{lemma}{Lemma}
\newtheorem{corollary}{Corollary}
\newtheorem{case}{Case}

\usepackage{amsthm}
\usepackage{lipsum}
\usepackage{tikz}

\makeatletter
\newcommand{\proofpart}[2]{%
  \par
  \addvspace{\medskipamount}%
  \noindent\emph{Part #1: #2}\par\nobreak
  \addvspace{\smallskipamount}%
  \@afterheading
}
\makeatother

\newcommand\restr[2]{{% we make the whole thing an ordinary symbol
  \left.\kern-\nulldelimiterspace % automatically resize the bar with \right
  #1 % the function
  \vphantom{\big|} % pretend it's a little taller at normal size
  \right|_{#2} % this is the delimiter
  }}

\theoremstyle{definition}
\newtheorem{definition}{Definition}
\newtheorem{remark}{Remark}

\usepackage{mathtools}
\DeclarePairedDelimiter\bra{\langle}{\rvert}
\DeclarePairedDelimiter\ket{\lvert}{\rangle}
\DeclarePairedDelimiterX\braket[2]{\langle}{\rangle}{#1 \delimsize\vert #2}

\usepackage{amsmath}
\usepackage{amsfonts}
\usepackage{amssymb}
\usepackage{fancyhdr}
\usepackage{tkz-euclide}

\DeclareMathOperator{\interior}{int}

\newcommand{\Tau}{\mathcal{T}}
\newcommand{\F}{\mathbb{F}}
\newcommand{\R}{\mathbb{R}}
\newcommand{\C}{\mathbb{C}}
\newcommand{\B}{\mathcal{B}}
\DeclarePairedDelimiterX{\iprod}[1]{\langle}{\rangle}{#1}
\DeclareMathOperator\rank{rank}

\newenvironment{amatrix}[1]{%
  \left(\begin{array}{@{}*{#1}{c}|c@{}}
}{%
  \end{array}\right)
}

\usepackage{calligra}
\DeclareMathAlphabet{\mathcalligra}{T1}{calligra}{m}{n}
\DeclareFontShape{T1}{calligra}{m}{n}{<->s*[2.2]callig15}{}

\newcommand{\scripty}[1]{\ensuremath{\mathcalligra{#1}}}

\pagestyle{fancy}
\author{Jeremiah Givens}
\newcommand{\subject}{Linear Algebra}
\newcommand{\Date}{9/2/2021} 
\makeatletter
\rhead{{\small\@author}}
\lhead{{\small\subject}}
\chead{{\large Homework 8}}
\cfoot{}
\rfoot{\thepage}
\lfoot{\today}

\renewcommand{\theequation}{\arabic{equation}}

\begin{document}
\section*{Problem 1}
Let $T$ be a linear operator on a finite dimensional vector space $V$ with Jordan Canonical Form 
\[\left[ 
\begin{array}{ccc|cc|cc}
2 &1 &0 &0 &0 &0 &0\\
0 &2 &1 &0 &0 &0 &0\\
0 &0 &2 &0 &0 &0 &0\\ \hline
0 &0 &0 &2 &1 &0 &0\\
0 &0 &0 &0 &2 &0 &0\\ \hline
0 &0 &0 &0 &0 &3 &0\\
0 &0 &0 &0 &0 &0 &3
\end{array}
\right]\]
Answer the following questions
    \begin{enumerate}
    \item Find the characteristic polynomial of $T$
    \item Find the dot diagram corresponding to each eigenvalue of $T$
    \item For which eigenvalues $\lambda_i$, if any, does $E_{\lambda_i} = K_{\lambda_i}$
    \item For each eigenvalue $\lambda_i$ find the smallest postive integer $p_i$ for which $K_{\lambda_i} = N((T_{\lambda_i} I)^{p_i})$
    \item Compute the following numbers for each $i$ where $U_i$ denotes the restriction of $T-\lambda_i$ to $K_{\lambda_i}$
        \begin{enumerate}
        \item $\rank(U_i)$
        \item $\rank(U_i^2)$
        \item $\dim(N(U_i))$
        \item $\dim(N(U_i^2))$
        \end{enumerate}
    \end{enumerate}
    
\subsection*{Solution}
\begin{enumerate}
\item The characteristic polynomial for $T$ is immediately obtainable from its Jordan Canonical Form:
\begin{align*}
f(t) = (2 - t)^5 (3 - \lambda)^2.
\end{align*}

\item For the eigenvalue $\lambda = 2$, we can see from our JCF of $T$ that we have one cycle of length $3$, and another cycle of length $2$. Thus, the dot diagram is
\begin{align*}
\begin{bmatrix}
\bullet & \bullet\\
\bullet & \bullet\\
\bullet & 
\end{bmatrix}.
\end{align*}
For the eigenvalue $\lambda = 3$, we have two cycles of length 1, so we have dot diagram
\begin{align*}
\begin{bmatrix}
\bullet & \bullet
\end{bmatrix}.
\end{align*}

\item For $\lambda_i = 3$ we have that $K_{\lambda_i} = E_{\lambda_i}$, as all of the vectors in $K_{\lambda_i}$ are eigenvalues of $T$.

\item These can immediately be obtained from the dot diagrams. If we let $\lambda_1 = 2$, and $\lambda_2 = 3$, we have
\begin{align*}
p_1 = 3 \text{, and } p_2 = 1.
\end{align*}

\item \begin{enumerate}
\item We have that $\rank(U_i)$ is equal to the number of dots in the dot diagram that are not the bottom dot for their column. Thus, we have
\begin{align*}
\rank(U_1) = 3 \text{, and } \rank(U_2) = 0.
\end{align*}

\item We have that $\rank(U_i^2)$ is equal to the number of dots in the dot diagram that are not in the bottom two dots for their column. Thus, 
\begin{align*}
\rank(U_1^2) = 1 \text{, and } \rank(U_2^2) = 0.
\end{align*}

\item The number of dots in the dot diagram is equal to the dimension of $K_{\lambda_i}$. Thus, using the rank nullity theorem, and our results from (a), we have
\begin{align*}
\dim N(U_1) = 5 - 3 = 2 \text{, and } \dim N(U_2) = 2 - 0 = 2.
\end{align*}

\item Using the results from part (b), we have
\begin{align*}
\dim N(U_1^2) = 5 - 1 = 4 \text{, and } \dim N(U_2^2) = 2 - 0 = 2.
\end{align*}
\end{enumerate}
\end{enumerate}

\section*{Problem 2}
Find the Jordan Canonical Form for the following matrix.
\[A = \begin{bmatrix}
1 &0 &2 &2\\
0 &1 &0 &0\\
0 &2 &3 &-2\\
0 &2 &2 &-1
\end{bmatrix}\]

\subsection*{Solution}
We start by finding the finding the characteristic polynomial for $A$:
\begin{align*}
|A - t I| &= \begin{vmatrix}
1 - t &0 &2 &2\\
0 &1 - t &0 &0\\
0 &2 &3 - t &-2\\
0 &2 &2 &-1 - t
\end{vmatrix}\\
&= (1 - t)^2 (-(3-t)(1+t) + 4)\\
&= (1 - t)^2(t^2 - 2t + 1)\\
&= (1-t)^4.
\end{align*}
Thus, we have one unique eigenvalue, $\lambda = 1$. Since this eigenvalue has multiplicity 4, we have $\dim (K_1) = 4$. We will now find the dot diagram, but finding how many dots go in each row:
\begin{align*}
r_1 &= \dim(V) - \rank (A - \lambda I)\\
&= 4 - \rank \left( \begin{bmatrix}
0 &0 &2 &2\\
0 &0 &0 &0\\
0 &2 &2 &-2\\
0 &2 &2 &-2
\end{bmatrix} \right)\\
&= 4 - \rank \left( \begin{bmatrix}
0 &2 &2 &-2\\
0 &0 &2 &2\\
0 &0 &0 & 0\\
0 &0 &0 & 0
\end{bmatrix} \right)\\
&= 4 - 2\\
&= 2
\end{align*}

Now, to find the number of dots in the second row, we will need to square $A - \lambda I$:
\begin{align*}
(A - \lambda I)^2 &= \begin{bmatrix}
0 &0 &2 &2\\
0 &0 &0 &0\\
0 &2 &2 &-2\\
0 &2 &2 &-2
\end{bmatrix}^2\\
&= \begin{bmatrix}
0 &0 &2 &2\\
0 &0 &0 &0\\
0 &2 &2 &-2\\
0 &2 &2 &-2
\end{bmatrix} \begin{bmatrix}
0 &0 &2 &2\\
0 &0 &0 &0\\
0 &2 &2 &-2\\
0 &2 &2 &-2
\end{bmatrix}\\
&= \begin{bmatrix}
0 &8 &8 &-8\\
0 &0 &0 &0\\
0 &0 &0 &0\\
0 &0 &0 &0
\end{bmatrix}.
\end{align*}
Thus, we have
\begin{align*}
r_2 &= r_1 - \rank (A - \lambda I)^2\\
&= 2 - 1\\
&= 1.
\end{align*}
Since the number of dots must equal the multiplicity of $\lambda$, we have $r_3 = 1$. Thus, the dot diagram is
\begin{align*}
\begin{bmatrix}
\bullet & \bullet\\
\bullet & \\
\bullet & 
\end{bmatrix}.
\end{align*}
With this, we can see that the JCF of $A$ is 
\begin{align*}
\begin{bmatrix}
1 &1 &0 &0\\
0 &1 &1 &0\\
0 &0 &1 &0\\
0 &0 &0 &1
\end{bmatrix}
\end{align*}

\section*{Problem 3}
Suppose that $V$ is the real vector space of functions spanned by the set of real-valued functions $\beta_0 = \{e^t, te^t, t^2e^t, e^{2t}\}$. Suppose that $T$ is the linear operator on $V$ defined by $T(f) = f'$. Find a Jordan Canonical Form $J$ of $T$ and a Jordan basis $\beta$ of generalized eigenvectors for $V$.

\subsection*{Solution}
We will start by expressing $T$ as a matrix in this given basis. We have
\begin{align*}
Te^t &= e^t\\
Tte^t &= e^t + te^t\\
Tt^2e^t &= 2te^t + t^2 e^t\\
Te^{2t} &= 2 e^{2t}.
\end{align*}
Thus, we have
\begin{align*}
[T]_{\beta_0} = \begin{bmatrix}
1 &1 &0 &0\\
0 &1 &2 &0\\
0 &0 &1 &0\\
0 &0 &0 &2
\end{bmatrix}.
\end{align*}

Now, we will find the characteristic polynomial:
\begin{align*}
|T - t I| &= \begin{vmatrix}
1 - t &1 &0 &0\\
0 &1 - t &2 &0\\
0 &0 &1 - t &0\\
0 &0 &0 &2 - t
\end{vmatrix}\\
&= (1-t)^3 (2 - t).
\end{align*}
Thus, we have two unique eigenvalues $\lambda_1 = 1$ and $\lambda_2 = 2$ with respective multiplicities $3$ and $1$. Now we will find matrix form of $T - \lambda_1 I$:
\begin{align*}
T - \lambda_1 I &= \begin{bmatrix}
1 - \lambda_1 &1 &0 &0\\
0 &1 - \lambda_1 &2 &0\\
0 &0 &1 - \lambda_1 &0\\
0 &0 &0 &2 - \lambda_1
\end{bmatrix}\\
&= \begin{bmatrix}
0 &1 &0 &0\\
0 &0 &2 &0\\
0 &0 &0 &0\\
0 &0 &0 &1
\end{bmatrix}.
\end{align*}
This has rank 3, which means that there is only 1 dot in the first row of our dot diagram, which means we can conclude that the dot diagram corresponding to $\lambda_1$ is
\begin{align*}
\begin{bmatrix}
\bullet\\
\bullet\\
\bullet
\end{bmatrix}.
\end{align*}
The dot diagram for $\lambda_2$ is trivial, thus we can conclude that a JCF of $T$ is
\begin{align*}
\begin{bmatrix}
1 &1 &0 &0\\
0 &1 &1 &0\\
0 &0 &1 &0\\
0 &0 &0 &2
\end{bmatrix}
\end{align*}

Now, we will find the nullspace of $(T - \lambda_1 I)^2$. We have
\begin{align*}
\begin{bmatrix}
0 &1 &0 &0\\
0 &0 &2 &0\\
0 &0 &0 &0\\
0 &0 &0 &1
\end{bmatrix}^2 &= \begin{bmatrix}
0 &1 &0 &0\\
0 &0 &2 &0\\
0 &0 &0 &0\\
0 &0 &0 &1
\end{bmatrix} \begin{bmatrix}
0 &1 &0 &0\\
0 &0 &2 &0\\
0 &0 &0 &0\\
0 &0 &0 &1
\end{bmatrix}\\
&= \begin{bmatrix}
0 &0 &2 &0\\
0 &0 &0 &0\\
0 &0 &0 &0\\
0 &0 &0 &1
\end{bmatrix}.
\end{align*}
Thus, the a basis for this null space is 
\begin{align*}
\left\{\begin{bmatrix}
1\\
0\\
0\\
0
\end{bmatrix}, \begin{bmatrix}
0\\
1\\
0\\
0
\end{bmatrix} \right\}.
\end{align*}
Now we find the null space for $(T - \lambda_1 I)^3$:
\begin{align*}
\begin{bmatrix}
0 &1 &0 &0\\
0 &0 &2 &0\\
0 &0 &0 &0\\
0 &0 &0 &1
\end{bmatrix}^3 &= \begin{bmatrix}
0 &1 &0 &0\\
0 &0 &2 &0\\
0 &0 &0 &0\\
0 &0 &0 &1
\end{bmatrix} \begin{bmatrix}
0 &0 &2 &0\\
0 &0 &0 &0\\
0 &0 &0 &0\\
0 &0 &0 &1
\end{bmatrix}\\
&= \begin{bmatrix}
0 &0 &0 &0\\
0 &0 &0 &0\\
0 &0 &0 &0\\
0 &0 &0 &1
\end{bmatrix}.
\end{align*}
Thus, we see that a basis for this null space is
\begin{align*}
\left\{\begin{bmatrix}
1\\
0\\
0\\
0
\end{bmatrix}, \begin{bmatrix}
0\\
1\\
0\\
0
\end{bmatrix},
\begin{bmatrix}
0\\
0\\
1\\
0
\end{bmatrix}\right\}.
\end{align*}

Now, we want the one vector that is not in the nullspace of $(T - \lambda I)^2$, so we know we want the third one. Call this vector $v_3$. We can use this to find the next vector in our basis that will put $T$ in JCF:
\begin{align*}
v_2 &= (T - \lambda I)v_3\\
&= Tt^2e^t - t^2e^t\\
&= 2te^t + t^2e^t - t^2 e^t\\
&= 2te^t.
\end{align*}
Finally, we use this to get the eigenvector:
\begin{align*}
v_1 &= (T - \lambda I)v_2\\
&= 2Tte^t - 2te^T\\
&= 2e^t + 2te^t - 2te^t\\
&= 2e^t.
\end{align*}
Putting it all together, our basis for putting $T$ into JCF is 
\begin{align*}
\beta = \{t^2e^t, 2te^t, 2e^t, e^{2t}\}.
\end{align*}

As a sanity check, we will express $T$ in this basis. First, we see
\begin{align*}
Tt^2 e^t &= t^2e^t + 2te^t\\
T2te^t &= 2te^t + 2e^t\\
T2e^t &= 2e^t\\
T2e^{2t} &= 2e^{2t}.
\end{align*}
With this, we can see
\begin{align*}
[T]_\beta = \begin{bmatrix}
1 &1 &0 &0\\
0 &1 &1 &0\\
0 &0 &1 &0\\
0 &0 &0 &2
\end{bmatrix},
\end{align*}
as desired.

\end{document}
