\documentclass[11 pt]{article}
\usepackage{geometry}
\geometry{letterpaper}
\usepackage{graphicx}
\usepackage{amssymb}
\usepackage{tikz}
\usetikzlibrary{decorations.pathreplacing}
\usetikzlibrary{arrows}
\usepackage{setspace}
\usepackage{amsmath}
\usepackage{fullpage}
\usepackage{amsthm}
\usepackage{enumitem}
\usepackage{mathtools}

\usepackage[colorlinks = true,
            linkcolor = blue,
            urlcolor  = blue,
            citecolor = blue,
            anchorcolor = blue]{hyperref}

\DeclareMathOperator\rowspan{Row}
\DeclareMathOperator\tr{Tr}
\DeclarePairedDelimiter\floor{\lfloor}{\rfloor}
\DeclareMathOperator\rank{rank}

\newcommand{\F}{\mathbb{F}}
\newcommand{\R}{\mathbb{R}}
\newcommand{\C}{\mathbb{C}}

\begin{document}
\onehalfspacing
\begin{center}
\textbf{{\Large Linear Algebra, Fall 2023}\\
Written Homework 4}\\
Due: 13 Sep 2023
\end{center}

\noindent Please answer the following questions on a separate sheet of paper and turn it in.

\begin{enumerate}
\item Let $A$ be a diagonalizable matrix. Prove that for any positive integer $m$, $A$ and $A^m$ have the same matrix $Q$ which diagonalizes them, and hence, the same basis of eigenvectors.

\item The trace of a matrix is defined to be the sum of the elements along the main diagonal, 
\[\tr(A) = \sum_i a_{ii}.\]
    \begin{enumerate}
    \item Given a complex monic polynomial,
    \[p(z) = z^n + a_{n-1}z^{n-1} + \cdots + a_1 z + a_0.\] show that $a_{n-1}$ is the sum of the complex roots of $p(z)$.
    \item Given a complex matrix $A \in M_{n\times n}(\C)$, with characteristic polynomial 
    \[g(z) = (-1)^n(z^n + a_{n-1}z^{n-1} + \cdots + a_1 z + a_0),\] Show that $-a_{n-1}$ is a sum of the eigenvalues (with multiplicity) of $A$.
    \item Use the fact that similar matrices have the same characteristic polynomial (you proved this in your last homework) to show that for a diagonalizable complex matrix $A$, 
    \[\tr(A) = -a_{n-1}.\]
    \end{enumerate}

\item Hertz Rent-a-Car has a fleet of 2000 cars distributed across three locations in the Huntsville area: 1) Airport, 2) Downtown Huntsville, and 3) Madison Blvd. The movement of cars from one location to another over the course of a week can be described as a Markov process with transition matrix
\[A = \begin{array}{c|c}
\text{Rented from:}\\
    \begin{array}{ccc} \text{Air} &\text{Dtwn} &\text{Mad}\\\hline
    .90 &.01 &.09\\
    .01 &.90 &.01\\
    .09 &.09 &.90
    \end{array} &\begin{array}{l} \text{Returned to:}\\\hline \text{Air}\\ \text{Dtwn} \\ \text{Mad} \end{array}
\end{array}\]
You may use a computer to help with completing these questions.
    \begin{enumerate}
    \item Find a diagonalizing matrix $Q$ and diagonal matrix $D$ so that $Q^{-1} A Q = D$
    \item What are the eigenvalues of the transition matrix?
    \item Find $\lim_{n \to \infty}A^n$.
    \item How many cars should be allocated at each location to minimize the numbers of cars which need to be moved from one lot to another at week's end?
    \end{enumerate}
\end{enumerate}
\end{document}